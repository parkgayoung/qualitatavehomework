\documentclass[american,man]{apa6}

\usepackage{amssymb,amsmath}
\usepackage{ifxetex,ifluatex}
\usepackage{fixltx2e} % provides \textsubscript
\ifnum 0\ifxetex 1\fi\ifluatex 1\fi=0 % if pdftex
  \usepackage[T1]{fontenc}
  \usepackage[utf8]{inputenc}
\else % if luatex or xelatex
  \ifxetex
    \usepackage{mathspec}
    \usepackage{xltxtra,xunicode}
  \else
    \usepackage{fontspec}
  \fi
  \defaultfontfeatures{Mapping=tex-text,Scale=MatchLowercase}
  \newcommand{\euro}{€}
\fi
% use upquote if available, for straight quotes in verbatim environments
\IfFileExists{upquote.sty}{\usepackage{upquote}}{}
% use microtype if available
\IfFileExists{microtype.sty}{\usepackage{microtype}}{}
\ifxetex
  \usepackage[setpagesize=false, % page size defined by xetex
              unicode=false, % unicode breaks when used with xetex
              xetex]{hyperref}
\else
  \usepackage[unicode=true]{hyperref}
\fi
\hypersetup{breaklinks=true,
            bookmarks=true,
            pdfauthor={Gayoung Park\^{}1},
            pdftitle={Qualitativeassignment},
            colorlinks=true,
            citecolor=blue,
            urlcolor=blue,
            linkcolor=magenta,
            pdfborder={0 0 0}}
\urlstyle{same}  % don't use monospace font for urls

\setlength{\emergencystretch}{3em}  % prevent overfull lines

\setcounter{secnumdepth}{0}
\ifxetex
  \usepackage{polyglossia}
  \setmainlanguage{}
\else
  \usepackage[american]{babel}
\fi

 % Line numbering
  \usepackage{lineno}
  \linenumbers


% Manuscript styling
\captionsetup{font=singlespacing,justification=justified}
\usepackage{csquotes}


% Essential manuscript parts
  \title{Qualitativeassignment}
  \shorttitle{Scientific explantion of starting point of Korean Upper Paleolithic}
  \author{Gayoung Park\(^1\)}
  \affiliation{\vspace{0.5cm}\(^1\)University of Washington}


\begin{document}

\maketitle

\subsection{Introduction}\label{introduction}

Most problematic and popular issue is about the starting point and
origin of upper Paleolithic period (MIS 3-2 transition). Some people
insist that emergence of blade (32,000 BP) is the starting point while
others think stemmed point (35,000 BP) indicate the start. And the
origin of upper Paleolithic had been regarded as coming from
Shuidonggou. However, nowadays this concept starts to be rethink about
the theory. The dates of the similar sites in Korea turned out being
earlier than Shuidonggou (Brantingham 2004; Brantingham et al. 2004;
Bar-Yosef 2007 all sited in Seong 2009). Therefore people find the new
origin and present three different models despite the huge gap of North
Korea (Bae 2012). First model is in situ evolutionary model which is
blade, stemmed point and other upper Paleolithic assemblages
autonomously emerged in the South of Korean peninsula. Second one is
migration model. From the north, not China, people came to the Korean
peninsula with new stone tool. Last one is traded and exchange model
mixing with migration one. Once people with new technology settled down
in the north part of Korea and then give their knowledge or lithics to
other group.

In this project, I would like to focus those three models and verify
each one based on archaeological evidences and links between evidences
and behaviors.

\subsection{Archaeological evidence}\label{archaeological-evidence}

blade/micro-blade technologies, stemmed points (new),traditional core
and flake, stratigraphic layer

Stone toolkits such as core and blade of Early Paleolithic continue to
be comprised of Late Paleolithic in Korea while other lithic assemblages
are comprised of blade and microblade ston tools (Bae 2012)

\subsection{Links between evidence and
behavior}\label{links-between-evidence-and-behavior}

migration or trade or both essential factors to illustrate modern human
behaviors:raw material availability, subsistence, and mobility systems

\subsection{Behavior at different
scales}\label{behavior-at-different-scales}

between indigenous foragers and non-indigenous among indigenous

\subsection{Discussion}\label{discussion}

Main reason of the debates is that the related Korean archaeological
records do not provide strongly perceived distinctive toolkit such as
Aurignacian in Europe (Lee 2013).

\begin{enumerate}
\def\labelenumi{\arabic{enumi})}
\itemsep1pt\parskip0pt\parsep0pt
\item
  Which model is more reasonable?
\end{enumerate}

(1)Seong: in situ, evolutionary- IBE \enquote{x-\textgreater{}y}, Salmon
\enquote{indigenous behavioral evolution} \enquote{The change of using
blade and blade tool is viewed as slow, evolutionary process that
eventually culimated in the Late Paleolithic transition}\\ \enquote{The
increasind frequency of blades in these sites is evidence for in situ
model} \enquote{Similar subsistene pattern: making stemmed point and
then microblade} His argument is based on \enquote{modern human
behavior} which I don't agree as Lee's opinion.(Seong 2009) He regards
the modern human behavior as \enquote{pattern}-world wide one. Then,
does it mean DN?/ontic? He applies the pattern into Korean case.. (He
didn't mention about modern human itself) -\textgreater{}generalized
strategy to a formalized technology in the Late Pleistocene can be
understood by the interplay of various factors, raw material
availability, subsistence, and mobility systems and so one. He thinks
that those changes can indicate evolution. And he concludes that the
case of Korean Upper Paleolithic is fit into the pattern. In addition,
Population density, growing intensity of competition among local
populations, complex site structure, and specializtion in animal
exploitation were changed in Upper Paleolithic (Gamble 1999; Gilman
1984; McBrearty and Brooks 2000; Whallon 1989, 2006 all sited in Seong
2009).(But Seoung doesn't mention about the exact evidence of increased
population.) He asserts that case of Asia is different with Europe, for
example, Mousterian failed to reach to East Asia. And that's why similar
tendency should be understood in the evolution approach emphasizing
\enquote{adaptation} . Main mechanism: climate changed -\textgreater{}
toolkits also changed ex) need for projectile technology -\textgreater{}
uneven resource distribution -\textgreater{} increasing mobility and
increasing social networks =\textgreater{} blade technology\\ (The
mechanism itself is causal..) How and why blades bacames so widespread?
= due to changes in settlement and subsistence systems, social
structure, and population dynamics. High mobility: formal/generalized
(standardized?) tools (and preparing core/blank which is suitable shape
and size) represent reduction of production cost and associated with
high monility since it minimizeds the weight of artifacts
hunter-gatherers need to carry. But I think this argument is not more
related with starting point but phase 2. (1)individual site in the
sequence of stratified sites -in a site: different characteristic
according to time differnece -difference between each site
=\textgreater{} high mobility -check all site-\textgreater{}IBE??

\begin{verbatim}
  (2)chronological sequence
  -no significant difference before 40,000
  -significannt difference between 40,000 and 30,000 : blades and SP were recovered at middle cultural horizons at Hwadae-ri, Hopyeong-dong, and Youngho-dong and these artifact types dominate the lithic assemblages from Yongsan-dong and Gorye-ri. (Which is same with my phases)
  
  (3)change in lithic assemblages
  -from earlier cobble- and pebble-tool-dominated assemblages to endscrapers, burins, and backed knives and produced from relatively high quality vein quartz. 
  -same kinds of tool such as choppers and polyhedrals: their size and frequency seem to have decreased
  
  (4)change in SP and blade assemblages
  -comparing the ratio: blade to flake ratio, large tool/small too ratio, and ratio of growing reliance on blade.-SP was kept using until microblade period (Jangheung-ri, Suyanggae, Jingeunul, Seockjang-ri, Sinbuk, and Wolpyeong).
  
  (5)Raw material change
  -use new material for blade and related lithics: fine-grained raw materials such as siliceous shale or siliceous tuff(from quartzite and vein quartz): often not available in certain location -> move or trade -> increasing mobility and increasing social networks
  -obsidian from Mt. Baekdu, Kyushu, and Hokkaido sources (Lee 2008 sited in Seong 2009)
\end{verbatim}

(2)Bae: migration, unification/ ontic \enquote{Combination of different
foraging groups emigrating from Siberia and southern China}
\enquote{Population movement} \enquote{Counter argument of in situ:
absence of continuous behavioral evolutionary transition such as from
Oldowan to Acheulean} \enquote{Genetic evidence: forgaing groups from
southern China (modern humans) that still used Early Paleolithic core
and flake tools migreaged northward to the Korean Peninsula}
-\textgreater{} question \enquote{Why did they move facing colder
environmental condition} -\textgreater{} answer: depending on
\enquote{Paleobathymetric variation : South china was dry and that
region and Korean Peninsula were connected at that time-swallow yellow
sea}, \enquote{Marine Isotope: not that much cold at that time}
\enquote{But still don't know why they moved}

Law like statement: modern people -\textgreater{} migrate
-\textgreater{} all modern people migrate -\textgreater{} it started
upper Paleolithic But it seems Salmon's causual!

Unification/(causality)

(3)Lee?: mixing model?/ (He says new technologies was introduced but
didn't change traditional assemblages) Basically His argument was
originated from errors of others'.--foil? The blade toolkits were
introduced, but did not immediately replaced pre/coexisting traditional
assemblage. Also blade technology did not replace the prexisiting
assemblage. Ful-fledged simple core and flake tool assemblages
(SCFA-pre/coexisting traditional assemblage) seems to reoccur around 100
ka and flourish unrtil 30ka. During the blade period, the SCFA exhibits
the general characteristics withouth a wide range of variation within
assemblage. Blade-based lithic technology initially starts around 35 ka
in Korea (Bae,2010). Evolutionary thoery (in situ) does not make sense
due to absence of any predetermined lithic strategies that require
extensive preparation, such as Levallois technique (Bar-Yosef and Kuhn
1999 sited in Lee 2013). And migration model should have Levallois
technique too because Levallois technology is comprised of not only the
Levallois technique, but also blades. The most possible region the blade
might originated from (migration thoery) is Altai region and 50ka (micro
blade-30ka) based on evidence from Denisova Cave (Derevianko 2011 sited
in Lee 2013). But there is huge time gap between Altai and Korea though
the initial period of blade introduction is still not clear in Korea.
The Korean blade assemblage from the period does not show sophisticated
balde technology.The number of case of blade core is limited. But, the
full-scale reduction sequence and crest technology exist in Korea (agree
with some part of migration model). There is natural barrier in Northern
part of Korean Penninsula, rugged mountain. Some blade-like assemblages
might be produced by accident. And there are numerous methods of
manufacturing blades or long flakes (Bar-Yosef and Kuhn 1999 sited in
Lee 2013). The blade technology in Korea is not related with Homo
spience(The age of the oldest one in Eurasia is younger than 40 ka).(The
analysis of hominin remains is practically impossible in Korea) DNA of
Denisovan from Altai are sisters to Neanderthals (Reich et al., 2010;
Meyer et al.,2012 sited in Lee 2013). Korean blades seem ti be the
result of a founder effect by pre-modern humans. (Maybe South-model of
Bae's assertion is possible.) He suggests the notion of \enquote{the
ancestor and descendant relationship}.

Migration/trade interaction model-modified version of migration model
\enquote{NO documented evidence of migration} \enquote{Small frequency
of new technology-blades and stemmed points at the beginning (slow
introduction) of the Late Plaeolithicm but still most of assemblage were
dominated by traditional lithics} \enquote{Possiblity of long distance
migration: obsidian, Arca shells}

IBE model?

\begin{enumerate}
\def\labelenumi{\arabic{enumi})}
\setcounter{enumi}{1}
\itemsep1pt\parskip0pt\parsep0pt
\item
  Application of scientific explanation
\end{enumerate}

hunter-gatherer's/Binford- functional explanation natural selection-
apply to functional explanation - such as evolutionary explanation- but
hard to verify/ unproveable. for example- no proxy in attribute of
lithic. but zoo archaeology- it works -markov-chain. community noum
(reading and check all readind and case and then determine whether or
not this case is general or is hard to explain in general way) and
practice informal-high probability

\subsection{Conclusion}\label{conclusion}

\subsection{Reference}\label{reference}

Seong, C.T., 2009. Emergence of a blade industry and evolution of Late
Paleolithic technology in the Republic of Korea. Journal of
Anthropological Research 65:417-451

Bae, K., 2010. Origin and pattern of the Upper Paleolithic industries in
the Korean Peninsula and movement of modern human in East Asia.
Quaternary International, 211(1-2): 103-112

Bae, C.J. \& Bae, K., 2012. The nature of the Early to Late Paleolithic
transition in Korea: Current perspectives. Quaternary International,
281: 26-35

Lee, H., 2013. Current observations of the early Late Paleolithic in
Korea. Quaternary International, Volume 316, pp.~45-58

\section{Note for writing}\label{note-for-writing}

\begin{enumerate}
\def\labelenumi{(\arabic{enumi})}
\item
  Modern human and blade/microblade So far, blade/microblade technology
  has been regarded as symbol of modern human. But current evidences
  support the counterargument of those relationship. In other words,
  there is some cases which illustrate no relationship between
  blade/microblade technology and human. For example, the technology got
  started ealier than evolution towards modern Homo sapiens in Africa.
  In addition, there is no blade/microblade eventhough modern human
  reached until Southeast Asia (Shea et al.,Bae \& Bae 2012).
\item
  the Korean Late Paleolithic can be divided into two cultural stages:
  1) an initial blade technology that appears sometime between 40 and
  36ka; 2)around 25ka microblade begin to appear
\end{enumerate}



\end{document}
