\documentclass[american,man]{apa6}

\usepackage{amssymb,amsmath}
\usepackage{ifxetex,ifluatex}
\usepackage{fixltx2e} % provides \textsubscript
\ifnum 0\ifxetex 1\fi\ifluatex 1\fi=0 % if pdftex
  \usepackage[T1]{fontenc}
  \usepackage[utf8]{inputenc}
\else % if luatex or xelatex
  \ifxetex
    \usepackage{mathspec}
    \usepackage{xltxtra,xunicode}
  \else
    \usepackage{fontspec}
  \fi
  \defaultfontfeatures{Mapping=tex-text,Scale=MatchLowercase}
  \newcommand{\euro}{€}
\fi
% use upquote if available, for straight quotes in verbatim environments
\IfFileExists{upquote.sty}{\usepackage{upquote}}{}
% use microtype if available
\IfFileExists{microtype.sty}{\usepackage{microtype}}{}
\ifxetex
  \usepackage[setpagesize=false, % page size defined by xetex
              unicode=false, % unicode breaks when used with xetex
              xetex]{hyperref}
\else
  \usepackage[unicode=true]{hyperref}
\fi
\hypersetup{breaklinks=true,
            bookmarks=true,
            pdfauthor={Gayoung Park\^{}1},
            pdftitle={Qualitativeassignment},
            colorlinks=true,
            citecolor=blue,
            urlcolor=blue,
            linkcolor=magenta,
            pdfborder={0 0 0}}
\urlstyle{same}  % don't use monospace font for urls

\setlength{\emergencystretch}{3em}  % prevent overfull lines

\setcounter{secnumdepth}{0}
\ifxetex
  \usepackage{polyglossia}
  \setmainlanguage{}
\else
  \usepackage[american]{babel}
\fi



% Manuscript styling
\captionsetup{font=singlespacing,justification=justified}
\usepackage{csquotes}


% Essential manuscript parts
  \title{Qualitativeassignment}
  \shorttitle{Scientific explantion of starting point of Korean Upper Paleolithic}
  \author{Gayoung Park\(^1\)}
  \affiliation{\vspace{0.5cm}\(^1\)University of Washington}


\begin{document}

\maketitle

\subsection{Introduction}\label{introduction}

The purpose of this paper is to reinterpret and evaluate existing
archaeological debate based on the approach of scientific explanation.
The target subject or the debate of this paper is about three different
models explaining the starting point and origin of Upper Paleolithic
period of Korea. I believe that scientific explanations, from Hempel to
IBE, play a role of sieve which strains partial and biased argument as
well as enable to do neutral and reasonable evaluation.

The debates of starting point and origin of Upper Paleolithic period
(MIS 3-2 transition) are most problematic and popular issue in the study
of Paleolithic in Korea. Some people insist that emergence of blade
(32,000 BP) is the starting point while others think stemmed point
(35,000 BP) indicate the start (K. Bae, 2010). And the origin of Upper
Paleolithic had been regarded as coming from Shuidonggou site located in
northeast China (Lee, 2013). However, this concept starts to be
reconsidered. Since the dates of the similar sites in Korea were turned
out be earlier than Shuidonggou (Seong, 2009). Therefore people have
tried to find the new origin, and Bae and Bae divide proposed models
into three different categories (C. J. Bae \& Bae, 2012). First model is
in situ evolutionary one which is that blade, stemmed point and other
Upper Paleolithic assemblages autonomously emerged in the South of
Korean peninsula. Second one is migration model. From the north, not
China but Siberia, and southern China, people came to the Korean
peninsula with new stone tools. Last model is trade and exchange model
mixing with migration one. Once people with new technology settled down
in the north part of Korea and then give their knowledge or lithics to
other group (C. J. Bae \& Bae, 2012).

In this project, I will analyze those three different models based on
archaeological perspectives as well as approaches of scientific
explanations and then evaluate the models.

The three models are introduced and supported in different articles: (1)
In situ model in an article, titled, \enquote{Emergence of a blade
industry and evolution of Late Paleolithic technology in the Republic of
Korea} (Seong, 2009), (2) Migration model in \enquote{Origin and pattern
of the Upper Paleolithic industries in the Korean Peninsula and movement
of modern human in East Asia} (K. Bae, 2010) and \enquote{The nature of
the Early to Late Paleolithic transition in Korea: Current perspectives}
(C. J. Bae \& Bae, 2012), and (3) Combination model in \enquote{Current
observations of the early Late Paleolithic in Korea} (Lee, 2013).

Even though the three authors do not mention about scientific approach
in the articles, I could find some of approaches such as Hempel's
unification (K. Bae, 2010), combination of plural approaches such as
Samon's causality and Hempel's unification (Seong, 2009), and Lipton's
IBE (Lee, 2013) in their assertions. Based on the explanations, I think
Lee's combination model is the most reasonable one which explains causes
of beginning of Upper Paleolithic in Korea.

\subsection{Archaeological evidence}\label{archaeological-evidence}

The main reason of the debates is that the related Korean archaeological
records do not provide strongly perceived distinctive toolkit such as
Aurignacian in Europe (Lee, 2013). Lithics, especially, blade industries
are the key component of the models. Blade and micro blades and related
tools such as stemmed points are regard as new technology of the period,
the starting point of Upper Paleolithic, however, majority of lithics,
core and flake tools, have been continuously used from the former
period. Usage of raw materials, change in lithic assemblage,
chronological sequence based on dating records, comparison with
different excavation sites, stratigraphic aspect, genetic analysis of
the Y-chromosome, and paleobathymetric variation, absence of Levallois
technique are also used for building model. More specific arguments are
followings:

\begin{enumerate}
\def\labelenumi{(\arabic{enumi})}
\item
  In situ model: Based on indigenous behavioral evolution, Seong asserts
  that the change of using blade and blade tool is viewed as slow,
  frequency of using the blade industries is increased, and similar
  pattern of making stemmed points and then microblades can be seen are
  indicating evolutionary processes gradually. The main mechanism of his
  evolutionary background is that climate change had driven needs of new
  toolkits including projectile points and it causes consideration or
  raw materials. And due to uneven distribution of resource or quarry,
  hunter-gatherers' mobility and social networks were increased and it
  derived blade technology. Seong compares lithic assemblages of
  different cultural layers in one site (time difference) as well as
  ones of different sites and infers with the results that different
  aspects of layers and sites indicates high mobility. If the aspect
  were common and stable, then it means that hunter-gatherers had no
  need to move. Chronological sequence shows that there was no
  significant change before 40,000BP but \enquote{'gradual}' changes,
  producing blades could be seen between 40,000 and 30,000BP at middle
  cultural horizons at Hwadae-ri, Hopyeong-dong, and Youngho-dong sites
  and these artifact types dominate the lithic assemblages from
  Yongsan-dong and Gorye-ri sites. From earlier cobble and pebble-tools
  such as choppers and polyhedrals dominated assemblages to endscrapers,
  burins, and backed knives and produced from relatively high quality
  vein quartz, the lithic assemblages were changed gradually. Not just
  for kinds of tools, ratios of blade to flake, large tool to small
  tool, and growing reliance on blade were also gradually changed
  (Seong, 2009).
\item
  Migration model: Bae's model is based on combination of different
  foraging groups emigrating from Siberia and southern China. He points
  out absence of continuous behavioral evolutionary transition such as
  from Oldowan to Acheulean. And he presents genetic studies and
  paleobathymetric variation. The analysis of Y-chromosomes shows
  relationship between modern human in Korean peninsula and in southern
  China. Foraging groups of southern China could easily go to Korean
  peninsula through route of Yellow Sea because the two regions were
  connected at that time. About low completeness of early blades and
  relating tools, he asserts that new comers might have adapted to the
  local environment in the Korean peninsula by adopting the conventional
  technology of tool production instead of retaining their own tool
  making tradition of producing blades. Proper raw material takes risk
  and cost. And southern China also had flake-based lithic industries.
  He asserts that the blade technology came from Denisovan, in
  southwestern Siberia according to similarity of lithics with Korean
  blades (K. Bae, 2010).
\item
  Combination model: This model can be called as migration-trade
  interaction model or modified version of migration model. This model
  was originated from errors of other models. Lee argues that new
  technologies were introduced in Korean peninsula but they didn't
  change traditional assemblages. In other words, the blade toolkits
  were introduced, but did not immediately replaced pre/coexisting
  traditional assemblage. The traditional lithic industry, or
  full-fledged simple core and flake tool assemblages (SCFA) seems to
  reoccur around 100 ka and flourish until 30ka. During the blade
  period, the SCFA exhibits the general characteristics without a wide
  range of variation within assemblage. Like Bae's assertion(K. Bae,
  2010), Lee thinks that evolutionary thoery (in situ) does not make
  sense due to absence of any predetermined lithic strategies that
  require extensive preparation, such as Levallois technique. But he
  also questions Bae's migration theory. Because there are rugged
  mountains as natural barrier in Northern part of Korean peninsula so
  it is hard to move in from north. In addition, the blade technology in
  Korea is not related with Homo spience which Bae thinks as foraging
  groups from southern China because the age of the oldest one in
  Eurasia is younger than 40 ka. It means that Homo spience, modern
  human might arrive in Korea much later (the analysis of hominine
  remains is practically impossible in Korea), but blade-technology
  based lithics was started before the period that the modern human
  arrived. However, he recognizes the possibility of migration in some
  point and trade interaction because of existence of obsidian and Arca
  shells which indicate long distance mobility (Lee, 2013).
\end{enumerate}

\subsection{Links between evidence and
behaviors}\label{links-between-evidence-and-behaviors}

The three models explain the same phenomenon in different ways such as
evolutionary theory, migration, and trade which are main frame of
explanations of each model. These evolution, migration, and trade are
broad range of human behavior and they contain small range of behaviors
such as adaptation for surrounding environment, mobility, or subsistence
pattern. Especially on this specific research period, the starting point
of Upper Paleolithic (MIS 3-2 transition), the term of \enquote{modern
human behavior} is frequently used to depict similar pattern of
subsistence, frequent mobility, and use law material.

The authors explain the introduction of incomplete blade toolkits and
coexistence with traditional assemblages as the result of modern human's
ability of adaptation (evolution) (Seong, 2009) or long-distance
mobility (K. Bae, 2010), or the result of combination with mobility and
trade (not by modern human) (Lee, 2013). Without the common evidence of
blade toolkit, Seong presents different aspect of sites and layer as
evidence of mobility, effort for adaptation, and in situ development
(Seong, 2009). Bae expects the migration from southern China and Siberia
with generic evidence of modern human on the basis of the fact that one
of main characteristic of modern human is highly mobile forager (K. Bae,
2010). Lee argues that the introduction of new technology by modern
human's migration or evolution is hard to believe due to the discordance
between dates of archaeological record and ones of Homo sapience.
Therefore he thinks that the new technology was the results of
combination of trade, some part of migration, and in situ-development
based on adaptation for endemic environment of Korean peninsula (Lee,
2013). Both Seong and Lee recognize the existence of social network or
mobility in Korean peninsula because of necessity of acquisition for
fine-grain raw material and actual evidences such as obsidian from Mt.
Baekdu, Kyushu, and Hokkaido sources (Lee, 2013).

\subsection{Behavior at different
scales}\label{behavior-at-different-scales}

Human behaviors depicted in the three models can be classified in two
ways: worldwide range and regional range. I already distinguished the
two ranges in the former section. For example, human's evolution and
migration are huge change in human history so these behaviors could
belong to worldwide range or broad range of human behavior. Small range
or regional range behaviors would be adaptation for surrounding
environment, trade, mobility, and subsistence pattern which lead to
producing the projectile points. Combination and accumulation of these
behaviors could become the worldwide range, getting into Upper
Paleolithic. And the evidences of the behavior could be detected in
archaeological data such as use of same raw material (Lee, 2013) and
genetic similarity (K. Bae, 2010).

In addition, I perceive other category of classification of human
behavior which is social network. On the basis of Korean peninsula,
there are two types of network could be existed: network between
indigenous foragers and outsiders and one among indigenous foragers.
Seong's model is closed to the latter type (Seong, 2009), Bae's to
former one (K. Bae, 2010), and Lee's to combination of the two types
(Lee, 2013).

\subsection{Explanatory models}\label{explanatory-models}

I do not think that Korean archaeologists explicitly employ methods of
scientific explanation (SE) into their models because no clear approach
of SE can be seen in their models. However, as most of archaeologists
do, they use methods of SE without even recognizing them. For example,
Seong's approach of the phenomenon of blade introduction seems to be
influenced by Binford and Binford. They criticize typological approach
like Bordes, focus on causations of assemblages and believe the
difference of lithics came from certain functional reason (L. R. Binford
\& Binford, 1966). Seong regards that the reason of introduction of new
lithic is for adaptation of changed environment which seems to
explanation of causality. \enquote{Climate change drives the new
lithics}. But I cannot conclude that Seong's model is only based on
causality or functionalism. Because he also understands the introduction
of new lithic with the worldwide range of modern human's behavior
pattern, evolution which seems approach of unification (Seong, 2009).

In the beginning of this project, I expected to classify the models
within Wylie's classification which divides type of explanation into
three: Epistemic theory of explanation, Ontic theories of explanation,
and Pragmatic or erotetic theories of explanations (Wylie, 1996).
However, like Seong's approach, I can recognize plural types of
explanation in one model.

Bae's migration model is quite clear to apply SE. His approach based on
common sense such as \enquote{modern human had significant
characteristic of long distance migration}. I think the he uses the
modern human's migration as \enquote{law-like} statement of Hempel and
Oppenheim's (Hempel \& Oppenheim, 1948) and applies this mechanism into
the case of Korean Upper Paleolithic (K. Bae, 2010).

Lee's model starts with questions about other's model. For example, even
though the geological condition was reasonable to migration (low sea
level), but Bae's model cannot explain a question \enquote{why} people
moved. And he verifies other's evidence such as dating of modern human
(Lee, 2013). I think his approach is similar with Glymour's or Hanon and
Kelly which stress the importance of test and verification of hypothesis
(Glymour, 1980). His combination model seems to the model of Inference
to the Best Explanation (IBE) which tries to provide best understanding
(Lipton, 2003).

\subsection{Explanatory model and relevant philosophy of science
literature.}\label{explanatory-model-and-relevant-philosophy-of-science-literature.}

As I mentioned in the previous chapter, Salmon's causality and Hempel's
DN approach, and Lipton's IBE can be seen in the three models. Here, I
would like to briefly introduce these explanatory approaches based on
Wylie's classification (Wylie, 1996) scrutinize the connection with
three models.

Among her classification, epistemic theory of explanation is a
\enquote{top-down} conception of explanation, and regarded as a function
of the systematizing power of theory. It includes models of
Hempel-Oppenheim, Friedman, and Kitcher. Hempel's model, in other words,
explanation by unification is the approach to explain by providing
unified accounts of wide rages of phenomena. For example, a number of
gas laws such as Boyle's law, Charles's law, Graham's law, etc can be
explained by Newtonian physics (Salmon, 1992). Bae regards the
characteristic of modern human, migration, as law (top or world-wide)
and applies directly into the region (down or regional) between Korean
peninsula and adjacent locations (K. Bae, 2010).

Ontic theory of explanation is a \enquote{bottom-up} approach, tries to
reveal the mechanisms based on causality. Salmon's model is included in
this category. Salmon (1992) illustrates that causal explanation is to
explain some phenomenon with finding and mentioning its cause (Salmon,
1992). Seong's mechanism is correlation between climate change and
lithics. But I think this mechanism also seems to be regarded as
law-statement in study of prehistory. In addition, he also applies more
apparent law, the evolution to modern human (Seong, 2009). This type of
combination model is not uncommon, especially the combination of Hempel
and Salmon as Seong does. It can support each limitation of the
approaches (Wylie, 1996).

Pragmatic or erotetic theory of explanations is a kind of family of
theories, and closed to why-question rather than law-like statement. Van
Frasseen asserts that scientific explanation is not pure science but an
application of science to describe certain phenomena. Therefore, it
should describe and explain individual cases and should be an answer of
why-question (Fraassen, 1980). From the Van Fraseen's point of view and
as Lee mentioned (Lee, 2013), Bae's migration model is not reasonable
because he cannot explain a lot of possibly derived questions such as
\enquote{why} the modern human in southern China came to Korean
peninsula (K. Bae, 2010).

In addition to Wylie's model, there is another approach, inference to
the Best Explanation (IBE). This approach also points out the weakness
of methods of Hempel and Salmon and focuses on individual case such as
pragmatic. IBE tries to find the most reasonable explanation, called
\enquote{loveliest explanation}. The difference with pragmatic approach
is more deepen why question such as why P \enquote{rather than} Q.
Interesting characteristic of IBE is process of testing hypothesis using
the notion of foil (Lipton, 2003). As Lipton did, Lee also points out
errors in other models based on archaeological evidence (Lee, 2013).
Especially he focuses on the weakness of application of law or law like
statement. Characteristic of modern human's migration or evolution does
not perfectly make sense in the case of Korean peninsula due to
discordance with dating of modern human skeleton and missing link of
middle stage between previous lithic industries and blades.

\subsection{Critique of three models}\label{critique-of-three-models}

To tell conclusion first, I think Lee's approach is the best explanation
among three models. As I mentioned in the previous section, he finds
errors in application the law, evolutionary theory and modern human's
migration, into case of Korea by applying or testing with new
archaeological records (Lee, 2013). Salmon shows as an example of the
limitation of Hempel's unification that law cannot cover some
specialties of each cases of archaeology such as functional explanation
in evolutionary theory (Salmon, 1992).

Except Lee's comments (Lee, 2013), in view of van Frassen (Fraassen,
1977), Seong misses the explanation why population was increased at that
time despite of severe climate change or the explanation of sequence in
his causal mechanism, so his adaption model is hard to accept (Seong,
2009). In addition, in view of Lipton (Lipton, 2003), Bae cannot explain
why (or how) the indigenous people in Korea accepted the new lithic
rather than refusing to use (K. Bae, 2010). However, as Barnes asserts,
I cannot judge whether Lee's loveliness explanation is the best one or
not (Barnes, 1995). But, then who knows the truth authentically? I think
the main role of archaeologist is pursuing most reasonable explanation
with continuously updated data. And based on current archaeological
records, Lee's combination model is the best and it also can be denied
or replaced in future.

\subsection{Conclusion}\label{conclusion}

The main task of this project is to reinterpret and evaluate three
models depicting starting point of Upper Paleolithic in Korean
peninsula. Each model has proper archaeological evidences and reflects
human behaviors in different scales. Even though, the authors seem not
recognize the approaches of scientific explanation, they still use the
scientific approach under influence of reading archaeologists, Renfrew
and Binford. I could succeed to find several kinds of scientific
explanations in their models and could evaluate each model.

As a result, I think that Lee's combination model is reasonable to
describe the period. Again, I do not know whether he considered the
scientific explanation as frame of his model or not, his approach is
quite reasonable. Testing (other's) hypotheses and recognizing foils of
them are important points to evaluate as well as to build a model.

\subsection*{Reference}\label{reference}
\addcontentsline{toc}{subsection}{Reference}

Bae, C. J., \& Bae, K. (2012). The nature of the Early to Late
Paleolithic transition in Korea: Current perspectives. \emph{Quaternary
International}, \emph{281}, 26--35.
doi:\href{http://dx.doi.org/10.1016/j.quaint.2011.08.044}{10.1016/j.quaint.2011.08.044}

Bae, K. (2010). Origin and patterns of the Upper Paleolithic industries
in the Korean Peninsula and movement of modern humans in East Asia.
\emph{Quaternary International}, \emph{211}(1?2), 103--112.
doi:\href{http://dx.doi.org/10.1016/j.quaint.2009.06.011}{10.1016/j.quaint.2009.06.011}

Barnes, E. (1995). Inference to the loveliest explanation.
\emph{Synthese}, \emph{103}(2), 251--277.
doi:\href{http://dx.doi.org/10.1007/BF01090049}{10.1007/BF01090049}

Binford, L. R., \& Binford, S. R. (1966). A Preliminary Analysis of
Functional Variability in the Mousterian of Levallois Facies.
\emph{American Anthropologist}, \emph{68}(2), 238--295.
doi:\href{http://dx.doi.org/10.1525/aa.1966.68.2.02a001030}{10.1525/aa.1966.68.2.02a001030}

Fraassen, B. C. V. (1977). The Pragmatics of Explanation. \emph{American
Philosophical Quarterly}, \emph{14}(2), 143--150. Retrieved from
\url{http://www.jstor.org/stable/20009661}

Fraassen, B. C. V. (1980). \emph{The Scientific Image}. Clarendon Press.

Glymour, C. (1980). Hypothetico-Deductivism Is Hopeless.
\emph{Philosophy of Science}, \emph{47}(2), 322--325. Retrieved from
\url{http://www.jstor.org/stable/187090}

Hempel, C. G., \& Oppenheim, P. (1948). Studies in the Logic of
Explanation. \emph{Philosophy of Science}, \emph{15}(2), 135--175.
Retrieved from \url{http://www.jstor.org/stable/185169}

Lee, H. W. (2013). Current observations of the early Late Paleolithic in
Korea. \emph{Quaternary International}, \emph{316}, 45--58.
doi:\href{http://dx.doi.org/10.1016/j.quaint.2013.03.025}{10.1016/j.quaint.2013.03.025}

Lipton. (2003). \emph{Inference to the Best Explanation}. Retrieved from
\url{https://books-google-com.offcampus.lib.washington.edu/books/about/Inference_to_the_Best_Explanation.html?hl=ko\&id=O52CAgAAQBAJ}

Salmon, W. C. (1992). Explanation in Archaeology: An Update. In L.
Embree (Ed.), \emph{Metaarchaeology} (pp. 243--253). Springer
Netherlands. Retrieved from
\url{http://link.springer.com.offcampus.lib.washington.edu/chapter/10.1007/978-94-011-1826-2_10}

Seong, C. (2009). Emergence of a Blade Industry and Evolution of Late
Paleolithic Technology in the Republic of Korea. \emph{Journal of
Anthropological Research}, \emph{65}(3), 417--451. Retrieved from
\url{http://www.jstor.org/stable/25608225}

Wylie, A. (1996). Unification and Convergence in Archaeological
Explanation: The Agricultural ?Wave-of-Advance? And the Origins of
Indo-European Languages. \emph{The Southern Journal of Philosophy},
\emph{34}(S1), 1--30.
doi:\href{http://dx.doi.org/10.1111/j.2041-6962.1996.tb00809.x}{10.1111/j.2041-6962.1996.tb00809.x}



\end{document}
