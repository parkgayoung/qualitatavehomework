\documentclass[american,man]{apa6}

\usepackage{amssymb,amsmath}
\usepackage{ifxetex,ifluatex}
\usepackage{fixltx2e} % provides \textsubscript
\ifnum 0\ifxetex 1\fi\ifluatex 1\fi=0 % if pdftex
  \usepackage[T1]{fontenc}
  \usepackage[utf8]{inputenc}
\else % if luatex or xelatex
  \ifxetex
    \usepackage{mathspec}
    \usepackage{xltxtra,xunicode}
  \else
    \usepackage{fontspec}
  \fi
  \defaultfontfeatures{Mapping=tex-text,Scale=MatchLowercase}
  \newcommand{\euro}{€}
\fi
% use upquote if available, for straight quotes in verbatim environments
\IfFileExists{upquote.sty}{\usepackage{upquote}}{}
% use microtype if available
\IfFileExists{microtype.sty}{\usepackage{microtype}}{}
\ifxetex
  \usepackage[setpagesize=false, % page size defined by xetex
              unicode=false, % unicode breaks when used with xetex
              xetex]{hyperref}
\else
  \usepackage[unicode=true]{hyperref}
\fi
\hypersetup{breaklinks=true,
            bookmarks=true,
            pdfauthor={Gayoung Park\^{}1},
            pdftitle={Qualitativeassignment},
            colorlinks=true,
            citecolor=blue,
            urlcolor=blue,
            linkcolor=magenta,
            pdfborder={0 0 0}}
\urlstyle{same}  % don't use monospace font for urls

\setlength{\emergencystretch}{3em}  % prevent overfull lines

\setcounter{secnumdepth}{0}
\ifxetex
  \usepackage{polyglossia}
  \setmainlanguage{}
\else
  \usepackage[american]{babel}
\fi

 % Line numbering
  \usepackage{lineno}
  \linenumbers


% Manuscript styling
\captionsetup{font=singlespacing,justification=justified}
\usepackage{csquotes}


% Essential manuscript parts
  \title{Qualitativeassignment}
  \shorttitle{Scientific explantion of starting point of Korean Upper Paleolithic}
  \author{Gayoung Park\(^1\)}
  \affiliation{\vspace{0.5cm}\(^1\)University of Washington}


\begin{document}

\maketitle

\subsection{Introduction}\label{introduction}

The purpose of this paper is to reinterpret and evaluate existing
archaeological debates based on the approach of scientific explanation.
The target subject of this paper is about three different models
explaining the starting point and origin of Upper Paleolithic period of
Korea. This subject is directly related to my own dissertation, the
study of stemmed points in Korean Peninsula. In the beginning of the
thesis, I should have to mention about the topic and take a position. I
believe that scientific approach enables neutral and reasonable
evaluation. For this reason, I choose the debates for this paper.

The debates of starting point and origin of Upper Paleolithic period
(MIS 3-2 transition) are most problematic and popular issue in the study
of Paleolithic in Korea. Some people insist that emergence of blade
(32,000 BP) is the starting point while others think stemmed point
(35,000 BP) indicate the start. And the origin of Upper Paleolithic had
been regarded as coming from Shuidonggou site located in northeast
China. However, nowadays this concept starts to be rethinking. Since the
dates of the similar sites in Korea were turned out be earlier than
Shuidonggou ({\textbf{???}}). Therefore people have tried to find the
new origin and they present the three different models. First model is
in situ evolutionary one which is that blade, stemmed point and other
Upper Paleolithic assemblages autonomously emerged in the South of
Korean peninsula. Second one is migration model. From the north, not
China but Siberia, and southern China, people came to the Korean
peninsula with new stone tools. Last model is trade and exchange model
mixing with migration one. Once people with new technology settled down
in the north part of Korea and then give their knowledge or lithics to
other group ({\textbf{???}} and Bae\_2012).

In this project, I would like to analyze those three different models
based on archaeological evidence, perspective of human behavior,
different scales of behaviors, apply several scientific approaches into
the models, and reinterpret and evaluate the models.

The three models are introduced and supported in different articles: (1)
In situ model in an article, titled, \enquote{Emergence of a blade
industry and evolution of Late Paleolithic technology in the Republic of
Korea} ({\textbf{???}}), (2) Migration model in \enquote{Origin and
pattern of the Upper Paleolithic industries in the Korean Peninsula and
movement of modern human in East Asia} ({\textbf{???}}) and \enquote{The
nature of the Early to Late Paleolithic transition in Korea: Current
perspectives} ({\textbf{???}} and Bae\_2012), and (3) Combination model
in \enquote{Current observations of the early Late Paleolithic in Korea}
({\textbf{???}}).

\subsection{Archaeological evidence}\label{archaeological-evidence}

Main reason of the debates is that the related Korean archaeological
records do not provide strongly perceived distinctive toolkit such as
Aurignacian in Europe ({\textbf{???}}). Lithics, especially, blade
industries are the key component of the models. Blade and micro blades
and related tools such as stemmed points are regard as new technology of
the period, the starting point of Upper Paleolithic, however, majority
of lithics, core and flake tools, have been continuously used from the
former period. Usage of law materials, change in lithic assemblage,
chronological sequence based on dating records, comparison with
different excavation sites, stratigraphic aspect, genetic analysis of
the Y-chromosome, and paleobathymetric variation, absence of Levallois
technique are also used for building model. More specific arguments are
followings:

\begin{enumerate}
\def\labelenumi{(\arabic{enumi})}
\item
  In situ model: Based on indigenous behavioral evolution, Seong asserts
  that the change of using blade and blade tool is viewed as slow,
  frequency of using the blade industries is increased, and similar
  pattern of making stemmed points and then microblades can be seen are
  indicating evolutionary processes gradually. The main mechanism of his
  evolutionary background is that climate change had driven needs of new
  toolkits including projectile points and it causes consideration or
  raw materials. And due to uneven distribution of resource or quarry,
  hunter-gatherers' mobility and social networks were increased and it
  derived blade technology. Seong compares lithic assemblages of
  different cultural layers in one site (time difference) as well as
  ones of different sites and infers with the results that different
  aspects of layers and sites indicates high mobility. If the aspect
  were common and stable, then it means that hunter-gatherers had no
  need to move. Chronological sequence shows that there was no
  significant change before 40,000BP but \enquote{'gradual}' changes,
  producing blades could be seen between 40,000 and 30,000BP at middle
  cultural horizons at Hwadae-ri, Hopyeong-dong, and Youngho-dong sites
  and these artifact types dominate the lithic assemblages from
  Yongsan-dong and Gorye-ri sites. From earlier cobble and pebble-tools
  such as choppers and polyhedrals dominated assemblages to endscrapers,
  burins, and backed knives and produced from relatively high quality
  vein quartz, the lithic assemblages were changed gradually. Not just
  for kinds of tools, ratios of blade to flake, large tool to small
  tool, and growing reliance on blade were also gradually changed
  ({\textbf{???}}).
\item
  Migration model: Bae's model is based on combination of different
  foraging groups emigrating from Siberia and southern China. He points
  out absence of continuous behavioral evolutionary transition such as
  from Oldowan to Acheulean. And he presents genetic studies and
  paleobathymetric variation. The analysis of Y-chromosomes shows
  relationship between modern human in Korean peninsula and in southern
  China. Foraging groups of southern China could easily go to Korean
  peninsula through route of yellow sea because the two regions were
  connected at that time. About low completeness of early blades and
  relating tools, he asserts that new comers might have adapted to the
  local environment in the Korean peninsula by adopting the conventional
  technology of tool production instead of retaining their own tool
  making tradition of producing blades. Proper raw material takes risk
  and cost. And southern China also had flake-based lithic industries.
  He asserts that the blade technology came from Denisovan, in
  southwestern Siberia according to similarity of lithics with Korean
  blades ({\textbf{???}}).
\item
  Combination model: This model can be called as migration-trade
  interaction model or modified version of migration model. This model
  was originated from errors of other models. Lee argues that new
  technologies were introduced in Korean peninsula but they didn't
  change traditional assemblages. In other words, the blade toolkits
  were introduced, but did not immediately replaced pre/coexisting
  traditional assemblage. The traditional lithic industry, or
  full-fledged simple core and flake tool assemblages (SCFA) seems to
  reoccur around 100 ka and flourish until 30ka. During the blade
  period, the SCFA exhibits the general characteristics without a wide
  range of variation within assemblage. Like Bae's
  assertion({\textbf{???}}), Lee thinks that evolutionary thoery (in
  situ) does not make sense due to absence of any predetermined lithic
  strategies that require extensive preparation, such as Levallois
  technique. But he also questions Bae's migration theory. Because there
  is rugged mountain as natural barrier in Northern part of Korean
  Penninsula so it is hard to move in from north. In addition, the blade
  technology in Korea is not related with Homo spience which Bae thinks
  as foraging groups from southern China because the age of the oldest
  one in Eurasia is younger than 40 ka. It means that Homo spience,
  modern human might arrive in Korea much later (the analysis of
  hominine remains is practically impossible in Korea), but
  blade-technology based lithics was started before the period that the
  modern human arrived. However, he recognizes the possibility of
  migration in some point and trade interaction because of existence of
  obsidian and Arca shells which indicate long distance mobility
  ({\textbf{???}}).
\end{enumerate}

\subsection{Links between evidence and
behavior}\label{links-between-evidence-and-behavior}

migration or trade or both essential factors to illustrate modern human
behaviors:raw material availability, subsistence, and mobility systems

\subsection{Behavior at different
scales}\label{behavior-at-different-scales}

between indigenous foragers and non-indigenous among indigenous

\subsection{Discussion}\label{discussion}

Main reason of the debates is that the related Korean archaeological
records do not provide strongly perceived distinctive toolkit such as
Aurignacian in Europe (Lee 2013).

\begin{enumerate}
\def\labelenumi{\arabic{enumi})}
\itemsep1pt\parskip0pt\parsep0pt
\item
  Which model is more reasonable?
\end{enumerate}

(1)Seong: in situ, evolutionary- IBE \enquote{x-\textgreater{}y}, Salmon
\enquote{indigenous behavioral evolution} \enquote{The change of using
blade and blade tool is viewed as slow, evolutionary process that
eventually culimated in the Late Paleolithic transition}\\ \enquote{The
increasind frequency of blades in these sites is evidence for in situ
model} \enquote{Similar subsistene pattern: making stemmed point and
then microblade} His argument is based on \enquote{modern human
behavior} which I don't agree as Lee's opinion.(Seong 2009) He regards
the modern human behavior as \enquote{pattern}-world wide one. Then,
does it mean DN?/ontic? He applies the pattern into Korean case.. (He
didn't mention about modern human itself) -\textgreater{}generalized
strategy to a formalized technology in the Late Pleistocene can be
understood by the interplay of various factors, raw material
availability, subsistence, and mobility systems and so one. He thinks
that those changes can indicate evolution. And he concludes that the
case of Korean Upper Paleolithic is fit into the pattern. In addition,
Population density, growing intensity of competition among local
populations, complex site structure, and specializtion in animal
exploitation were changed in Upper Paleolithic (Gamble 1999; Gilman
1984; McBrearty and Brooks 2000; Whallon 1989, 2006 all sited in Seong
2009).(But Seoung doesn't mention about the exact evidence of increased
population.) He asserts that case of Asia is different with Europe, for
example, Mousterian failed to reach to East Asia. And that's why similar
tendency should be understood in the evolution approach emphasizing
\enquote{adaptation} . Main mechanism: climate changed -\textgreater{}
toolkits also changed ex) need for projectile technology -\textgreater{}
uneven resource distribution -\textgreater{} increasing mobility and
increasing social networks =\textgreater{} blade technology\\ (The
mechanism itself is causal..) How and why blades bacames so widespread?
= due to changes in settlement and subsistence systems, social
structure, and population dynamics. High mobility: formal/generalized
(standardized?) tools (and preparing core/blank which is suitable shape
and size) represent reduction of production cost and associated with
high monility since it minimizeds the weight of artifacts
hunter-gatherers need to carry. But I think this argument is not more
related with starting point but phase 2. (1)individual site in the
sequence of stratified sites -in a site: different characteristic
according to time differnece -difference between each site
=\textgreater{} high mobility -check all site-\textgreater{}IBE??

\begin{verbatim}
  (2)chronological sequence
  -no significant difference before 40,000
  -significannt difference between 40,000 and 30,000 : blades and SP were recovered at middle cultural horizons at Hwadae-ri, Hopyeong-dong, and Youngho-dong and these artifact types dominate the lithic assemblages from Yongsan-dong and Gorye-ri. (Which is same with my phases)
  
  (3)change in lithic assemblages
  -from earlier cobble- and pebble-tool-dominated assemblages to endscrapers, burins, and backed knives and produced from relatively high quality vein quartz. 
  -same kinds of tool such as choppers and polyhedrals: their size and frequency seem to have decreased
  
  (4)change in SP and blade assemblages
  -comparing the ratio: blade to flake ratio, large tool/small too ratio, and ratio of growing reliance on blade.-SP was kept using until microblade period (Jangheung-ri, Suyanggae, Jingeunul, Seockjang-ri, Sinbuk, and Wolpyeong).
  
  (5)Raw material change
  -use new material for blade and related lithics: fine-grained raw materials such as siliceous shale or siliceous tuff(from quartzite and vein quartz): often not available in certain location -> move or trade -> increasing mobility and increasing social networks
  -obsidian from Mt. Baekdu, Kyushu, and Hokkaido sources (Lee 2008 sited in Seong 2009)
  
   analogic model derive the power of problem-solving from comparison
   I agree with Kohler and Leeuw (2007).-focus on context itself.
   
   
   fine-grain/coarse-grain
\end{verbatim}

(2)Bae: migration, unification/ ontic \enquote{Combination of different
foraging groups emigrating from Siberia and southern China}
\enquote{Population movement} \enquote{Counter argument of in situ:
absence of continuous behavioral evolutionary transition such as from
Oldowan to Acheulean} \enquote{Genetic evidence: forgaing groups from
southern China (modern humans) that still used Early Paleolithic core
and flake tools migreaged northward to the Korean Peninsula}
-\textgreater{} question \enquote{Why did they move facing colder
environmental condition} -\textgreater{} answer: depending on
\enquote{Paleobathymetric variation : South china was dry and that
region and Korean Peninsula were connected at that time-swallow yellow
sea}, \enquote{Marine Isotope: not that much cold at that time}
\enquote{But still don't know why they moved}

Law like statement: modern people -\textgreater{} migrate
-\textgreater{} all modern people migrate -\textgreater{} it started
upper Paleolithic But it seems Salmon's causual!

Unification/(causality)

(3)Lee?: mixing model?/ (He says new technologies was introduced but
didn't change traditional assemblages) Basically His argument was
originated from errors of others'.--foil? The blade toolkits were
introduced, but did not immediately replaced pre/coexisting traditional
assemblage. Also blade technology did not replace the prexisiting
assemblage. Ful-fledged simple core and flake tool assemblages
(SCFA-pre/coexisting traditional assemblage) seems to reoccur around 100
ka and flourish unrtil 30ka. During the blade period, the SCFA exhibits
the general characteristics withouth a wide range of variation within
assemblage. Blade-based lithic technology initially starts around 35 ka
in Korea (Bae,2010). Evolutionary thoery (in situ) does not make sense
due to absence of any predetermined lithic strategies that require
extensive preparation, such as Levallois technique (Bar-Yosef and Kuhn
1999 sited in Lee 2013). And migration model should have Levallois
technique too because Levallois technology is comprised of not only the
Levallois technique, but also blades. The most possible region the blade
might originated from (migration thoery) is Altai region and 50ka (micro
blade-30ka) based on evidence from Denisova Cave (Derevianko 2011 sited
in Lee 2013). But there is huge time gap between Altai and Korea though
the initial period of blade introduction is still not clear in Korea.
The Korean blade assemblage from the period does not show sophisticated
balde technology.The number of case of blade core is limited. But, the
full-scale reduction sequence and crest technology exist in Korea (agree
with some part of migration model). There is natural barrier in Northern
part of Korean Penninsula, rugged mountain. Some blade-like assemblages
might be produced by accident. And there are numerous methods of
manufacturing blades or long flakes (Bar-Yosef and Kuhn 1999 sited in
Lee 2013). The blade technology in Korea is not related with Homo
spience(The age of the oldest one in Eurasia is younger than 40 ka).(The
analysis of hominin remains is practically impossible in Korea) DNA of
Denisovan from Altai are sisters to Neanderthals (Reich et al., 2010;
Meyer et al.,2012 sited in Lee 2013). Korean blades seem ti be the
result of a founder effect by pre-modern humans. (Maybe South-model of
Bae's assertion is possible.) He suggests the notion of \enquote{the
ancestor and descendant relationship}.

Migration/trade interaction model-modified version of migration model
\enquote{NO documented evidence of migration} \enquote{Small frequency
of new technology-blades and stemmed points at the beginning (slow
introduction) of the Late Plaeolithicm but still most of assemblage were
dominated by traditional lithics} \enquote{Possiblity of long distance
migration: obsidian, Arca shells}

IBE model?

\begin{enumerate}
\def\labelenumi{\arabic{enumi})}
\setcounter{enumi}{1}
\itemsep1pt\parskip0pt\parsep0pt
\item
  Application of scientific explanation
\end{enumerate}

hunter-gatherer's/Binford- functional explanation natural selection-
apply to functional explanation - such as evolutionary explanation- but
hard to verify/ unproveable. for example- no proxy in attribute of
lithic. but zoo archaeology- it works -markov-chain. community noum
(reading and check all readind and case and then determine whether or
not this case is general or is hard to explain in general way) and
practice informal-high probability

\subsection{Conclusion}\label{conclusion}

\section{Note for writing}\label{note-for-writing}

\begin{enumerate}
\def\labelenumi{(\arabic{enumi})}
\item
  Modern human and blade/microblade So far, blade/microblade technology
  has been regarded as symbol of modern human. But current evidences
  support the counterargument of those relationship. In other words,
  there is some cases which illustrate no relationship between
  blade/microblade technology and human. For example, the technology got
  started ealier than evolution towards modern Homo sapiens in Africa.
  In addition, there is no blade/microblade eventhough modern human
  reached until Southeast Asia (Shea et al.,Bae \& Bae 2012). (Lim et
  al., 2007)
\item
  the Korean Late Paleolithic can be divided into two cultural stages:
  1) an initial blade technology that appears sometime between 40 and
  36ka; 2)around 25ka mcroblade begin to appear
\end{enumerate}

Lim, H. S., Lee, Y. I., Yi, S., Kim, C.-B., Chung, C.-H., Lee, H.-J., \&
Choi, J. H. (2007). Vertebrate burrows in late Pleistocene paleosols at
Korean Palaeolithic sites and their significance as a stratigraphic
marker. \emph{Quaternary Research}, \emph{68}(2), 213--219.
doi:\href{http://dx.doi.org/10.1016/j.yqres.2007.05.001}{10.1016/j.yqres.2007.05.001}



\end{document}
