\documentclass[american,man]{apa6}

\usepackage{amssymb,amsmath}
\usepackage{ifxetex,ifluatex}
\usepackage{fixltx2e} % provides \textsubscript
\ifnum 0\ifxetex 1\fi\ifluatex 1\fi=0 % if pdftex
  \usepackage[T1]{fontenc}
  \usepackage[utf8]{inputenc}
\else % if luatex or xelatex
  \ifxetex
    \usepackage{mathspec}
    \usepackage{xltxtra,xunicode}
  \else
    \usepackage{fontspec}
  \fi
  \defaultfontfeatures{Mapping=tex-text,Scale=MatchLowercase}
  \newcommand{\euro}{€}
\fi
% use upquote if available, for straight quotes in verbatim environments
\IfFileExists{upquote.sty}{\usepackage{upquote}}{}
% use microtype if available
\IfFileExists{microtype.sty}{\usepackage{microtype}}{}
\ifxetex
  \usepackage[setpagesize=false, % page size defined by xetex
              unicode=false, % unicode breaks when used with xetex
              xetex]{hyperref}
\else
  \usepackage[unicode=true]{hyperref}
\fi
\hypersetup{breaklinks=true,
            bookmarks=true,
            pdfauthor={Gayoung Park\^{}1},
            pdftitle={Qualitativeassignment},
            colorlinks=true,
            citecolor=blue,
            urlcolor=blue,
            linkcolor=magenta,
            pdfborder={0 0 0}}
\urlstyle{same}  % don't use monospace font for urls

\setlength{\emergencystretch}{3em}  % prevent overfull lines

\setcounter{secnumdepth}{0}
\ifxetex
  \usepackage{polyglossia}
  \setmainlanguage{}
\else
  \usepackage[american]{babel}
\fi



% Manuscript styling
\captionsetup{font=singlespacing,justification=justified}
\usepackage{csquotes}


% Essential manuscript parts
  \title{Qualitativeassignment}
  \shorttitle{Scientific explantion of starting point of Korean Upper Paleolithic}
  \author{Gayoung Park\(^1\)}
  \affiliation{\vspace{0.5cm}\(^1\)University of Washington}


\begin{document}

\maketitle

\subsection{Introduction}\label{introduction}

The purpose of this paper is to reinterpret and evaluate existing
archaeological debates based on the approach of scientific explanation.
The target subject of this paper is about three different models
explaining the starting point and origin of Upper Paleolithic period of
Korea. This subject is directly related to my own dissertation, the
study of stemmed points in Korean Peninsula. In the beginning of the
thesis, I should have to mention about the topic and take a position. I
believe that scientific approach enables neutral and reasonable
evaluation. For this reason, I choose the debates for this paper.

The debates of starting point and origin of Upper Paleolithic period
(MIS 3-2 transition) are most problematic and popular issue in the study
of Paleolithic in Korea. Some people insist that emergence of blade
(32,000 BP) is the starting point while others think stemmed point
(35,000 BP) indicate the start. And the origin of Upper Paleolithic had
been regarded as coming from Shuidonggou site located in northeast
China. However, nowadays this concept starts to be rethinking. Since the
dates of the similar sites in Korea were turned out be earlier than
Shuidonggou ({\textbf{???}}). Therefore people have tried to find the
new origin and they present the three different models. First model is
in situ evolutionary one which is that blade, stemmed point and other
Upper Paleolithic assemblages autonomously emerged in the South of
Korean peninsula. Second one is migration model. From the north, not
China but Siberia, and southern China, people came to the Korean
peninsula with new stone tools. Last model is trade and exchange model
mixing with migration one. Once people with new technology settled down
in the north part of Korea and then give their knowledge or lithics to
other group ({\textbf{???}}).

In this project, I would like to analyze those three different models
based on archaeological evidence, perspective of human behavior,
different scales of behaviors, apply several scientific approaches into
the models, and reinterpret and evaluate the models.

The three models are introduced and supported in different articles: (1)
In situ model in an article, titled, \enquote{Emergence of a blade
industry and evolution of Late Paleolithic technology in the Republic of
Korea} ({\textbf{???}}), (2) Migration model in \enquote{Origin and
pattern of the Upper Paleolithic industries in the Korean Peninsula and
movement of modern human in East Asia} ({\textbf{???}}) and \enquote{The
nature of the Early to Late Paleolithic transition in Korea: Current
perspectives} ({\textbf{???}}), and (3) Combination model in
\enquote{Current observations of the early Late Paleolithic in Korea}
({\textbf{???}}).

\subsection{Archaeological evidence}\label{archaeological-evidence}

Main reason of the debates is that the related Korean archaeological
records do not provide strongly perceived distinctive toolkit such as
Aurignacian in Europe ({\textbf{???}}). Lithics, especially, blade
industries are the key component of the models. Blade and micro blades
and related tools such as stemmed points are regard as new technology of
the period, the starting point of Upper Paleolithic, however, majority
of lithics, core and flake tools, have been continuously used from the
former period. Usage of law materials, change in lithic assemblage,
chronological sequence based on dating records, comparison with
different excavation sites, stratigraphic aspect, genetic analysis of
the Y-chromosome, and paleobathymetric variation, absence of Levallois
technique are also used for building model. More specific arguments are
followings:

\begin{enumerate}
\def\labelenumi{(\arabic{enumi})}
\item
  In situ model: Based on indigenous behavioral evolution, Seong asserts
  that the change of using blade and blade tool is viewed as slow,
  frequency of using the blade industries is increased, and similar
  pattern of making stemmed points and then microblades can be seen are
  indicating evolutionary processes gradually. The main mechanism of his
  evolutionary background is that climate change had driven needs of new
  toolkits including projectile points and it causes consideration or
  raw materials. And due to uneven distribution of resource or quarry,
  hunter-gatherers' mobility and social networks were increased and it
  derived blade technology. Seong compares lithic assemblages of
  different cultural layers in one site (time difference) as well as
  ones of different sites and infers with the results that different
  aspects of layers and sites indicates high mobility. If the aspect
  were common and stable, then it means that hunter-gatherers had no
  need to move. Chronological sequence shows that there was no
  significant change before 40,000BP but \enquote{'gradual}' changes,
  producing blades could be seen between 40,000 and 30,000BP at middle
  cultural horizons at Hwadae-ri, Hopyeong-dong, and Youngho-dong sites
  and these artifact types dominate the lithic assemblages from
  Yongsan-dong and Gorye-ri sites. From earlier cobble and pebble-tools
  such as choppers and polyhedrals dominated assemblages to endscrapers,
  burins, and backed knives and produced from relatively high quality
  vein quartz, the lithic assemblages were changed gradually. Not just
  for kinds of tools, ratios of blade to flake, large tool to small
  tool, and growing reliance on blade were also gradually changed
  ({\textbf{???}}).
\item
  Migration model: Bae's model is based on combination of different
  foraging groups emigrating from Siberia and southern China. He points
  out absence of continuous behavioral evolutionary transition such as
  from Oldowan to Acheulean. And he presents genetic studies and
  paleobathymetric variation. The analysis of Y-chromosomes shows
  relationship between modern human in Korean peninsula and in southern
  China. Foraging groups of southern China could easily go to Korean
  peninsula through route of yellow sea because the two regions were
  connected at that time. About low completeness of early blades and
  relating tools, he asserts that new comers might have adapted to the
  local environment in the Korean peninsula by adopting the conventional
  technology of tool production instead of retaining their own tool
  making tradition of producing blades. Proper raw material takes risk
  and cost. And southern China also had flake-based lithic industries.
  He asserts that the blade technology came from Denisovan, in
  southwestern Siberia according to similarity of lithics with Korean
  blades ({\textbf{???}}).
\item
  Combination model: This model can be called as migration-trade
  interaction model or modified version of migration model. This model
  was originated from errors of other models. Lee argues that new
  technologies were introduced in Korean peninsula but they didn't
  change traditional assemblages. In other words, the blade toolkits
  were introduced, but did not immediately replaced pre/coexisting
  traditional assemblage. The traditional lithic industry, or
  full-fledged simple core and flake tool assemblages (SCFA) seems to
  reoccur around 100 ka and flourish until 30ka. During the blade
  period, the SCFA exhibits the general characteristics without a wide
  range of variation within assemblage. Like Bae's
  assertion({\textbf{???}}), Lee thinks that evolutionary thoery (in
  situ) does not make sense due to absence of any predetermined lithic
  strategies that require extensive preparation, such as Levallois
  technique. But he also questions Bae's migration theory. Because there
  is rugged mountain as natural barrier in Northern part of Korean
  Penninsula so it is hard to move in from north. In addition, the blade
  technology in Korea is not related with Homo spience which Bae thinks
  as foraging groups from southern China because the age of the oldest
  one in Eurasia is younger than 40 ka. It means that Homo spience,
  modern human might arrive in Korea much later (the analysis of
  hominine remains is practically impossible in Korea), but
  blade-technology based lithics was started before the period that the
  modern human arrived. However, he recognizes the possibility of
  migration in some point and trade interaction because of existence of
  obsidian and Arca shells which indicate long distance mobility
  ({\textbf{???}}).
\end{enumerate}

\subsection{Links between evidence and
behavior}\label{links-between-evidence-and-behavior}

About the advent of new technology, in other words, blade technology,
the three models explain the same phenomenon in different ways.
Evolutionary theory, migration, and trade are main frame of explanations
of each model. These evolution, migration, and trade are broad range of
human behavior and they contain small range of behaviors such as
adaptation for surrounding environment, mobility, or subsistence
pattern. Especially on this specific research period, the starting point
of Upper Paleolithic (MIS 3-2 transition), the term of \enquote{modern
human behavior} is frequently used to depict similar pattern of
subsistence, frequent mobility, and use law material.

Interesting point is that the three models explain introduction of
incomplete blade toolkits and coexistence with traditional assemblages
as the result of modern human's ability of adaptation (evolution)
({\textbf{???}}) or long-distance mobility ({\textbf{???}}), or the
result of combination with mobility and trade (not by modern human)
({\textbf{???}}). Without the common evidence of blade toolkit, Seong
presents different aspect of sites and layer as evidence of mobility,
effort for adaptation, and in situ development ({\textbf{???}}). Bae
expects the migration from southern China and Siberia with generic
evidence of modern human on the basis of the fact that one of main
characteristic of modern human is highly mobile forager
({\textbf{???}}). Lee argues that the introduction of new technology by
modern human's migration or evolution is hard to believe due to the
discordance between dates of archaeological record and ones of Homo
sapience. Therefore he thinks that the new technology was the results of
combination of trade, some part of migration, and in situ-development
based on adaptation for endemic environment of Korean peninsula
({\textbf{???}}). Both Seong and Lee recognize the existence of social
network or mobility in Korean peninsula because of necessity of
acquisition for fine-grain raw material and actual evidences such as
obsidian from Mt. Baekdu, Kyushu, and Hokkaido sources ({\textbf{???}}).

\subsection{Behavior at different
scales}\label{behavior-at-different-scales}

Human behaviors depicted in the three models can be classified in two
ways: worldwide range and regional range. I already distinguished the
two ranges in the former chapter. For example, human's evolution and
migration are huge change in human history so these behaviors could
belong to worldwide range or broad range of human behavior. Small range
or regional range behaviors would be adaptation for surrounding
environment, trade, mobility, and subsistence pattern. Combination and
accumulation of these behaviors could become the worldwide range. And
the evidences of the behavior could be detected in archaeological data
such as use of same raw material ({\textbf{???}}) and genetic similarity
({\textbf{???}}).

In addition, I perceive other category of classification of human
behavior which is social network. On the basis of Korean peninsula,
there are two types of network could be existed: network between
indigenous foragers and outsiders and one among indigenous foragers.
Seong's model is closed to the latter type ({\textbf{???}}), Bae's to
former one ({\textbf{???}}), and Lee's to combination of the two types
({\textbf{???}}).

\subsection{Explanatory model}\label{explanatory-model}

I do not think that Korean archaeologists employ methods of scientific
explanation (SE) into their models because no clear approach of SE can
be seen in their models. However, as most of archaeologists do, they use
methods of SE without even recognizing them. For example, Seong's
approach of the phenomenon of blade introduction seems to be influenced
by Binford and Binford. They criticize typological approach like Bordes,
focus on causations of assemblages and believe the difference of lithics
came from certain functional reason ({\textbf{???}}). Seong regards that
the reason of introduction of new lithic is for adaptation of changed
environment. But I cannot conclude that Seong's model is only based on
causality or functionalism. Because he also understands the introduction
of new lithic with the worldwide range of modern human's behavior
pattern, evolution ({\textbf{???}}). In addition, he scrutinizes all
relating sites, considers each characteristic, and tries to find more
proper explanation. I think this approach is similar with pragmatism
({\textbf{???}}) or Lipton's loveliest explanation ({\textbf{???}}).

In the beginning of this project, I expected to classify the models
within Wylie's classification which divides type of explanation into
three: Epistemic theory of explanation, Ontic theories of explanation,
and Pragmatic or erotetic theories of explanations ({\textbf{???}}).
However, like Seong's approach, I can recognize plural types of
explanation in one model.

Bae's migration model is quite clear to apply SE. His approach based on
theory of modern human's migration is similar with using
\enquote{law-like} statement of Hempel and Oppenheim's ({\textbf{???}}).
His main argue is that modern human migrate and it starts Upper
Paleolithic so the case of Korea is not different ({\textbf{???}}).

Lee's model starts with questions about other's model. For example, even
though the geological condition was reasonable to migration (low sea
level), but Bae's model cannot explain a question \enquote{why} people
moved. And he verifies other's evidence such as dating of modern human
({\textbf{???}}). I think his approach is similar with Glymour's or
Hanon and Kelly which stress the importance of test and verification of
hypothesis ({\textbf{???}}). His combination model seems like the model
of Inference to the Best Explanation (IBE) which tries to provide best
understanding ({\textbf{???}}).

\subsection{Explanatory model and relevant philosophy of science
literature.}\label{explanatory-model-and-relevant-philosophy-of-science-literature.}

As I mentioned in the previous chapter, Salmon's causality, Hempel's DN
approach, van Fraassen's pragmatism, combination of plural models, and
Lipton's IBE can be seen in the three models. Here, I would like to
briefly introduce these explanatory model based on Wylie's
classification ({\textbf{???}}).

Among her classification, epistemic theory of explanation is a
\enquote{top-down} conception of explanation, and regarded as a function
of the systematizing power of theory. It includes models of
Hempel-Oppenheim, Friedman, and Kitcher. Hempel's model, in other words,
explanation by unification is the approach to explain by providing
unified accounts of wide rages of phenomena. For example, a number of
gas laws such as Boyle's law, Charles's law, Graham's law, etc can be
explained by Newtonian physics. The explanation of unification uses the
notion of covering law, which new archaeology arisen from, based on the
concept that every bona fide explanation makes essential reference to at
least one law of nature. But this approach has some limitation that the
method of unification often contains the reduction of one domain of
science and the reduction in the behavior science is more problematic.
For instance, the law cannot cover some specialties of each cases of
archaeology such as functional explanation in evolutionary theory
({\textbf{???}}).

Ontic theory of explanation is a \enquote{bottom-up} approach, tries to
reveal the mechanisms based on causality. Salmon's model is included in
this category. Salmon (1992) illustrates that causal explanation is to
explain some phenomenon with finding and mentioning its cause. Its
application into archaeological case needs to be taken in conjunction
with recognition of the basic statistical character explanations. For
instance, hunting strategy can be explained as a kind of ratio of
yielding success (behavior science). More direct approach of causal
explanation is to develop a theory of probabilistic causality. But it is
not a simple, to make the theory valid, both contributory causes and
counteracting causes should correspond to each other ({\textbf{???}}).
However, Salmon's approach also cannot explain individual facts. There
are some cases that the request for explanation is rejected and
asymmetry revealed by the barometer ({\textbf{???}}).

Pragmatic or erotetic theory of explanations is a kind of family of
theories, and closed to why-question rather than law-like statement. Van
Frasseen asserts that scientific explanation is not pure science but an
application of science to describe certain phenomena. Therefore, it
should describe and explain individual cases and should be an answer of
why-question ({\textbf{???}}).

In addition to Wylie's model, there is another approach, inference to
the Best Explanation (IBE). This approach also points out the weakness
of methods of Hempel and Salmon and focuses on individual case such as
pragmatic. IBE tries to find the most reasonable explanation, called
\enquote{loveliest explanation}. The difference with pragmatic approach
is more deepen why question such as why P \enquote{rather than} Q.
Interesting characteristic of IBE is process of testing hypothesis using
the notion of foil ({\textbf{???}}). However, there are problems of
judgments. We cannot judge whether loveliness explanation is really
better likeliness one or not, and whether loveliness explanation can be
matched with likeliness one ({\textbf{???}}).

Besides, several combination models are suggested. Mostly they suggest
combine two explanations of Hempel and Salmon but they place high
importance on Salmon's. For example, Wylie agrees with advantage of
Kitcher's unification and Renfrew's model and asserts that those
approaches need ontic explanation ({\textbf{???}}). Strevens also tries
to combine those two approaches into one unified model named the
kairetic account ({\textbf{???}}). And Kuznar and Long say that
archaeological explanations must use a combination of specific causal
mechanisms and generalizable laws (Kuznar and Long, 2008).

\subsection{Critique of three models}\label{critique-of-three-models}

To tell conclusion first, I think Lee's approach is the best explanation
among three models. As his opinion, test of hypotheses, there is no
evidence of existence modern human at the starting point of Upper
Paleolithic in Korean peninsula ({\textbf{???}}). Seong misses the
explanation why population was increased at that time despite of severe
climate change or the explanation of sequence in his causal mechanism,
so his adaption model is hard to accept ({\textbf{???}}). Bae cannot
explain why foraging group of southern China moved to Korea rather
staying in China ({\textbf{???}}). For these reasons, I prefer to Lee's
approach, the combination model.

They overlooked delicate time gap between introduction of stemmed points
(35,000 BP) and blades (32,000 BP). Stemmed points started earlier than
blades and I think this time gap can explain the phenomenon as well as
support in situ model. However, I don't think that blade was also made
by indigenous people. This technology was use in more broad geological
range so that migration or trade is also reasonable in some point,
definitely after stemmed points. Therefore, I think Upper Paleolithic
was started by indigenous people but the concept of blade technology
came from outside of Korea.

\subsection{Conclusion}\label{conclusion}

The main task of this project is to reinterpret and evaluate three
models depicting starting point of Upper Paleolithic in Korean
peninsula. Each model has proper archaeological evidences and reflects
human behaviors in different scales. Even though, the authors seem not
recognize the approaches of scientific explanation, they still use the
scientific approach under influence of reading archaeologists, Renfrew
and Binford. I could succeed to find several kinds of scientific
explanations in their models and could evaluate each model.

As a result, I think that Lee's combination model is reasonable to
describe the period. Again, I do not know whether he considered the
scientific explanation as frame of his model or not, his approach is
quite reasonable. Testing (other's) hypotheses and recognizing foils of
them are important points to evaluate as well as to build a model.

\subsection*{Reference}\label{reference}
\addcontentsline{toc}{subsection}{Reference}

Kuznar and Long. (2008). \emph{Against the Grain}. Retrieved from
\url{https://books-google-com.offcampus.lib.washington.edu/books/about/Against_the_Grain.html?hl=ko\&id=H-TINNqkqPIC}



\end{document}
